% --------------------------------------------------------------
% This is all preamble stuff that you don't have to worry about.
% Head down to where it says "Start here"
% --------------------------------------------------------------

\documentclass[12pt]{article}

\usepackage[margin=1in]{geometry}
\usepackage{amsmath,amsthm,amssymb}
\usepackage[margin=1in]{geometry}
\usepackage{amsmath,amsthm,amssymb}
\usepackage[french]{babel} %Castellanización
\usepackage[T1]{fontenc} %escribe lo del teclado
\usepackage[utf8]{inputenc} %Reconoce algunos símbolos
\usepackage{lmodern} %optimiza algunas fuentes
\usepackage{graphicx}
\graphicspath{ {images/} }
\usepackage{hyperref} % Uso de links
\usepackage[utf8]{inputenc}
\usepackage{enumitem}
\usepackage{braket}
\usepackage[makeroom]{cancel}
\def\p{\ensuremath\partial}
\newcommand{\N}{\mathbb{N}}
\newcommand{\Z}{\mathbb{Z}}
\newcommand{\R}{\mathbb{R}}
\usepackage{mathtools}
\usepackage{booktabs} % Required for better horizontal rules in tables
\usepackage{float}
\usepackage[utf8]{inputenc}


\newenvironment{theorem}[2][Theorem]{\begin{trivlist}
\item[\hskip \labelsep {\bfseries #1}\hskip \labelsep {\bfseries #2.}]}{\end{trivlist}}
\newenvironment{lemma}[2][Lemma]{\begin{trivlist}
\item[\hskip \labelsep {\bfseries #1}\hskip \labelsep {\bfseries #2.}]}{\end{trivlist}}
\newenvironment{exercise}[2][Exercise]{\begin{trivlist}
\item[\hskip \labelsep {\bfseries #1}\hskip \labelsep {\bfseries #2.}]}{\end{trivlist}}
\newenvironment{problem}[2][Problem]{\begin{trivlist}
\item[\hskip \labelsep {\bfseries #1}\hskip \labelsep {\bfseries #2.}]}{\end{trivlist}}
\newenvironment{question}[2][Question]{\begin{trivlist}
\item[\hskip \labelsep {\bfseries #1}\hskip \labelsep {\bfseries #2.}]}{\end{trivlist}}
\newenvironment{corollary}[2][Corollary]{\begin{trivlist}
\item[\hskip \labelsep {\bfseries #1}\hskip \labelsep {\bfseries #2.}]}{\end{trivlist}}

\newenvironment{solution}{\begin{proof}[Solution]}{\end{proof}}

\newcommand{\der}[2]{\frac{\text{d} #1}{\text{d} #2}}
\newcommand{\pder}[2]{\frac{\partial #1}{\partial #2}}
\newcommand{\dif}[1]{\text{d}#1}
\newcommand{\Tr}[1]{\mathrm{Tr}\hspace{1pt}\left[#1\right]}
\newcommand{\bdot}[0]{\boldsymbol{\cdot}}


\begin{document}
%\tableofcontents
\title{PHY6650 - Fond. théo. du modèle standard - Résumé}
\author{Olivier Denis et Guillaume Laurin}
\maketitle
\newpage

\section{Matrices de Pauli}
\begin{equation}
    \sigma_1 = \begin{pmatrix} 0 & 1\\ 1 & 0\\ \end{pmatrix} \quad , \quad \sigma_2 = \begin{pmatrix} 0  & -i \\ i & 0\\ \end{pmatrix} \quad , \quad \sigma_3 = \begin{pmatrix} 1 & 0 \\ 0 & -1 \\ \end{pmatrix}
\end{equation}

\section{Identités de matrice $\gamma$}
p.31\\
\begin{equation}
  \gamma_0 = \gamma^0 = \begin{pmatrix}
    1 & 0 \\
    0 & -1\\
\end{pmatrix} \text{  et   } \gamma^i = \begin{pmatrix}
  0 & \sigma_i \\
  -\sigma_i & 0\\
\end{pmatrix} \quad \implies \quad \{\gamma^\mu, \gamma^\nu\} = 2 g^{\mu\nu}
\end{equation}
\begin{equation}
  \gamma^{\mu\dag} = \gamma^0 \gamma^\mu \gamma^0 = ( \gamma^0; -\gamma^i )
\end{equation}
p.41 \\
\begin{equation}
  \overline{\psi} = \psi^\dag\gamma_0
\end{equation}

p.42 \\
\begin{equation}
  \gamma_5 = \gamma^5 = i\gamma^0\gamma^1\gamma^2\gamma^3\gamma^4 = \frac{i}{4!}\varepsilon_{\mu\nu\rho\sigma}\gamma^\mu\gamma^\nu\gamma^\rho\gamma^\sigma=\begin{pmatrix}0 & 1 \\ 1 & 0\\
\end{pmatrix}
\end{equation}
\begin{equation}
  \sigma^{\mu\nu} = \{\gamma^\mu,\gamma^\nu\}
\end{equation}

Devoir 1 \\
\begin{align}
  \gamma^\mu\gamma_\mu &= 4 \\
  \gamma^\mu\gamma^\nu\gamma_\mu &= -2\gamma^\nu \\
  \gamma^\mu\gamma^\nu\gamma^\rho\gamma_\mu &= 4g^{\nu\rho}\\
  \gamma^\mu\gamma^\nu\gamma^\rho\gamma^\sigma\gamma_\mu &= -2\gamma^\sigma\gamma^\rho\gamma^\nu\\
  \gamma^\mu\gamma^\nu\gamma^\rho\gamma^\sigma\gamma^\tau\gamma_\mu &= 2(\gamma^\tau\gamma^\nu\gamma^\rho\gamma^\sigma + \gamma^\sigma\gamma^\rho\gamma^\nu\gamma^\tau)
\end{align}

p.113-114 \\
\begin{equation}
  \Tr{\gamma^\mu\gamma^\nu} = 4g^{\mu\nu}
\end{equation}
\begin{equation}
  \Tr{\cancel{p}_1...\cancel{p}_n} = 0 \quad \forall\, n\, \in \, 2\Z+1
\end{equation}
\begin{equation}
  \Tr{\cancel{A}\cancel{B}\cancel{C}\cancel{D}} = 4(A\bdot B C \bdot D - A \bdot C B \bdot D + A \bdot D B \bdot C )
\end{equation}

p.146 \\
\begin{equation}
  \Tr{\gamma^\mu} = 0
\end{equation}
\begin{equation}
  \Tr{\gamma_5\cancel{p}_1...\cancel{p}_n} = 0 \quad \forall\, n\, \in \, 2\Z+1
\end{equation}
\begin{equation}
  \Tr{\gamma_5\gamma^\mu\gamma^\nu} = 0
\end{equation}
\begin{equation}
  \Tr{\gamma_5\gamma^\alpha\gamma^\beta\gamma^\gamma\gamma^\delta} = -4i\varepsilon^{\alpha\beta\gamma\delta}
\end{equation}

p.141-142
\begin{equation}
  \gamma_L = \frac{1}{2}(1-\gamma_5) \quad \implies \quad \gamma_L^2 = \gamma_L
\end{equation}
\begin{equation}
  \gamma_R = \frac{1}{2}(1+\gamma_5) \quad \implies \quad \gamma_R^2 = \gamma_R
\end{equation}
\begin{equation}
  \psi_L = \gamma_L\psi \quad \implies \quad \overline{\psi}_L = \overline{\psi}\gamma_R
\end{equation}
\begin{equation}
  \psi_R = \gamma_R\psi \quad \implies \quad \overline{\psi}_R = \overline{\psi}\gamma_L
\end{equation}

p.255 \\
\begin{equation}
  \psi^c = C\overline{\psi}^T \quad \text{où} \quad C = i\gamma^0\gamma^2
\end{equation}
\begin{equation}
  \overline{\psi^c} = 
\end{equation}
\begin{equation}
  (\psi_L)^c = \gamma_R\psi^c \quad \implies \quad \overline{(\psi_L)^c} = \overline{\psi^c}\gamma_L
\end{equation}

\section{Variables de Mandelstam}
p. 118

Pour un processus $(p, p') \rightarrow (k, k')$ :
\begin{align}
 E_{CM}^2 =  s &= (p + p')^2 = (k + k')^2 \\
  t &= (k - p)^2 = (k' - p')^2 \\
  u &= (k - p')^2 = (k' - p)^2
\end{align}

\begin{equation}
  \Rightarrow s+t+u = \sum_{i=1}^4 m_i^2
\end{equation}

\section{Section efficace}
p. 128

Pour une interaction $\alpha (1,2) \rightarrow \beta (3,.., n)$ :
\begin{align}
d\sigma_{\beta\alpha} = \frac{(2\pi)^4 \delta^{4}(p_\beta - p_\alpha)}{2 \sqrt{s - (m_1 + m_2)^2} \sqrt{s - (m_1 - m_2)^2}} |\mathcal{M_{\beta\alpha}}|^2 \prod_{i=3}^n \frac{d^3p_i}{2E_i (2\pi)^3} \prod_{k}^{types} \frac{1}{n_k!}
\end{align}

Pour une désintégration $\alpha (1) \rightarrow \beta (2,..., n)$ :
\begin{align}
d\Gamma_{\beta\alpha} = \frac{(2\pi)^4 \delta^4 (p_\beta - p_\alpha)}{2E_0} |\mathcal{M_{\beta\alpha}}|^2 \prod_{i=2}^n {\frac{d^3 p_i}{2 E_i (2\pi)^3}} \prod_{k}^{types} \frac{1}{n_k!}
\end{align}

\section{Taux de désintégration calculés en classe}
...

\section{Règles de Feynman pour l'amplitude d'interaction}

\hfill \break
Les règles de Feynman pour les lignes fermioniques sont (p. 108) :

\begin{itemize}
  \item création d'un fermion : $\overline{u}$
  \item annihilation d'un fermion : $u$
  \item création d'un anti-fermion : $v$
  \item annihilation d'un anti-fermion : $\overline{v}$
\end{itemize}

\hfill \break
Les termes de vertex sont (p. 108 et devoir 6) :

\begin{itemize}
  \item $ff\gamma$ : $-ie\gamma^\mu$
  \item $ffW$ : $\displaystyle \frac{ig}{\sqrt{2}} \gamma_\mu \gamma_L$
  \item $ffZ$ : $\displaystyle \frac{ig}{2 \cos{\theta}_W} \gamma_\mu \gamma_L$
  \item $Z(q) W^+(p) W^-(k)$ (avec $p + k + q = 0$) : $ig\cos{\theta}_W [ g_{\mu\nu}(k-p)_\alpha +  g_{\nu\alpha}(k-p)_\mu +  g_{\alpha\mu}(k-p)_\nu]$
\end{itemize} 


\hfill \break
Les propagateurs sont :

\begin{itemize}
  \item Fermion (p. 86) : $\displaystyle \frac{i(\cancel{p} + m)}{p^2 - m^2}$
  \item Photon (p. 94) : $\displaystyle \frac{-i g_{\mu\nu}}{p^2}$
  \item W et Z (p. 167 et devoir 6) : $\displaystyle \frac{-i(g_{\mu\nu} - p^\mu p^\nu/M^2)}{p^2-M^2}$
\end{itemize}

\hfill \break
Chaque boson créé nécessite aussi un terme de polarisation : $\epsilon_\mu$

\subsection{Notes sur les calculs de section efficace}

p. 123

\begin{equation}
  d^3 k = |\vec{k}| E_k dE_k d\Omega
\end{equation}

\begin{equation}
  \int f(x) \delta(g(x)) dx = \sum_i^{g(x_i)=0} \frac{f(x_i)}{|g'(x_i)|}
\end{equation}

\section{Équations fondamentales}
Convention : $c=\hbar=1$ \\

\textbf{Équation de Klein-Gordon} (p.25)
\begin{equation}
    \left[\frac{1}{c^2}\frac{\partial^2}{\partial t^2} - \Vec{\nabla}^2 + \left(\frac{mc}{\hbar}\right)\right]\phi = 0 \quad \iff \quad [\Box + m] \phi = 0
\end{equation}

où $\Box = \partial_\mu\partial^\mu$.\\

\textbf{Équation énergie-masse-impulsion} (p.25)
\begin{equation}
    E^2 = \Vec{p}^2 + m^2
\end{equation}

\textbf{Équation de Dirac} (p.30)
\begin{equation}
    (\cancel{p} -m )\psi = 0
\end{equation}

\textbf{Équation de Dirac conjuguée} (Devoir 3 Question 2)
\begin{equation}
    \overline{\psi}(\cancel{p} -m ) = 0
\end{equation}


\section{Spins et polarisations}
p.53
\begin{align}
    \sum_s u(p,s)\Bar{u}(p,s) &= (\cancel{p}+m)\\
    \sum_s v(p,s)\Bar{v}(p,s) &= (\cancel{p}-m)
\end{align}

p.133 \\
\begin{align}
  \sum_{polarisation} \varepsilon_\mu^*\varepsilon_\nu &= - g_{\mu\nu} \quad \text{(photon)}\\
  \sum_{polarisation} \varepsilon_\mu^*\varepsilon_\nu &= - \left[g_{\mu\nu}- \frac{P_\mu P_\nu}{P^2}\right] \quad \text{(boson W)}
\end{align}

\section{Chiralité et hélicité}
\begin{align}
    u(p)^+ &= \begin{bmatrix}
      \begin{pmatrix}
          1\\ 0
      \end{pmatrix}\\
      \begin{pmatrix}
          1\\ 0
      \end{pmatrix}
    \end{bmatrix}
    \quad \text{(hélicité droite)}\\
    u(p)^- &= \begin{bmatrix}
      \begin{pmatrix}
          0\\1
      \end{pmatrix}\\
      -\begin{pmatrix}
          0\\1
      \end{pmatrix}
    \end{bmatrix}
    \quad \text{(hélicité gauche)}\\
    \phi &=
    \begin{pmatrix}
          1\\ 0
     \end{pmatrix}
     \quad \text{(chiralité droite)}\\
     \phi &=
    \begin{pmatrix}
          0\\1
      \end{pmatrix}
      \quad \text{(chiralité gauche)}
\end{align}

Les particules sans masses sont d'hélicité et de chiralité droite. Similairement, les antiparticules sans masses sont d'hélicité et de chiralité gauche.


\end{document}
