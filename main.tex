% --------------------------------------------------------------
% This is all preamble stuff that you don't have to worry about.
% Head down to where it says "Start here"
% --------------------------------------------------------------

\documentclass[12pt]{article}

\usepackage[margin=1in]{geometry}
\usepackage{amsmath,amsthm,amssymb}
\usepackage[margin=1in]{geometry}
\usepackage{amsmath,amsthm,amssymb}
\usepackage[french]{babel} %Castellanización
\usepackage[T1]{fontenc} %escribe lo del teclado
\usepackage[utf8]{inputenc} %Reconoce algunos símbolos
\usepackage{lmodern} %optimiza algunas fuentes
\usepackage{graphicx}
\graphicspath{ {images/} }
\usepackage{hyperref} % Uso de links
\usepackage[utf8]{inputenc}
\usepackage{enumitem}
\usepackage{braket}
\usepackage[makeroom]{cancel}
\def\p{\ensuremath\partial}
\newcommand{\N}{\mathbb{N}}
\newcommand{\Z}{\mathbb{Z}}
\newcommand{\R}{\mathbb{R}}
\usepackage{mathtools}
\usepackage{booktabs} % Required for better horizontal rules in tables
\usepackage{float}
\usepackage[utf8]{inputenc}


\newenvironment{theorem}[2][Theorem]{\begin{trivlist}
\item[\hskip \labelsep {\bfseries #1}\hskip \labelsep {\bfseries #2.}]}{\end{trivlist}}
\newenvironment{lemma}[2][Lemma]{\begin{trivlist}
\item[\hskip \labelsep {\bfseries #1}\hskip \labelsep {\bfseries #2.}]}{\end{trivlist}}
\newenvironment{exercise}[2][Exercise]{\begin{trivlist}
\item[\hskip \labelsep {\bfseries #1}\hskip \labelsep {\bfseries #2.}]}{\end{trivlist}}
\newenvironment{problem}[2][Problem]{\begin{trivlist}
\item[\hskip \labelsep {\bfseries #1}\hskip \labelsep {\bfseries #2.}]}{\end{trivlist}}
\newenvironment{question}[2][Question]{\begin{trivlist}
\item[\hskip \labelsep {\bfseries #1}\hskip \labelsep {\bfseries #2.}]}{\end{trivlist}}
\newenvironment{corollary}[2][Corollary]{\begin{trivlist}
\item[\hskip \labelsep {\bfseries #1}\hskip \labelsep {\bfseries #2.}]}{\end{trivlist}}

\newenvironment{solution}{\begin{proof}[Solution]}{\end{proof}}

\newcommand{\der}[2]{\frac{\text{d} #1}{\text{d} #2}}
\newcommand{\pder}[2]{\frac{\partial #1}{\partial #2}}
\newcommand{\dif}[1]{\text{d}#1}
\newcommand{\Tr}[1]{\mathrm{Tr}\hspace{1pt}\left[#1\right]}
\newcommand{\bdot}[0]{\boldsymbol{\cdot}}


\begin{document}
%\tableofcontents
\title{PHY6650 - Fond. théo. du modèle standard - Résumé}
\author{Olivier Denis et Guillaume Laurin}
\maketitle
\newpage

\section{Identités de matrice $\gamma$}
p.31\\
\begin{equation}
  \gamma_0 = \gamma^0 = \begin{pmatrix}
    1 & 0 \\
    0 & -1\\
\end{pmatrix} \text{  et   } \gamma^i = \begin{pmatrix}
  0 & \sigma_i \\
  -\sigma_i & 0\\
\end{pmatrix} \quad \implies \quad \{\gamma^\mu, \gamma^\nu\} = 2 g^{\mu\nu}
\end{equation}

p.41 \\
\begin{equation}
  \overline{\psi} = \psi\gamma_0
\end{equation}

p.42 \\
\begin{equation}
  \gamma_5 = \gamma^5 = i\gamma^0\gamma^1\gamma^2\gamma^3\gamma^4 = \frac{i}{4!}\varepsilon_{\mu\nu\rho\sigma}\gamma^\mu\gamma^\nu\gamma^\rho\gamma^\sigma=\begin{pmatrix}0 & 1 \\ 1 & 0\\
\end{pmatrix}
\end{equation}
\begin{equation}
  \sigma^{\mu\nu} = \{\gamma^\mu,\gamma^\nu\}
\end{equation}

Devoir 1 \\
\begin{equation}
  \gamma^\mu\gamma_\mu = 4
\end{equation}
\begin{equation}
  \gamma^\mu\gamma^\nu\gamma_\mu = -2\gamma^\nu
\end{equation}
\begin{equation}
  \gamma^\mu\gamma^\nu\gamma^\rho\gamma_\mu = 4g^{\nu\rho}
\end{equation}
\begin{equation}
  \gamma^\mu\gamma^\nu\gamma^\rho\gamma^\sigma\gamma_\mu = -2\gamma^\sigma\gamma^\rho\gamma^\nu
\end{equation}
\begin{equation}
  \gamma^\mu\gamma^\nu\gamma^\rho\gamma^\sigma\gamma^\tau\gamma_\mu = 2(\gamma^\tau\gamma^\nu\gamma^\rho\gamma^\sigma + \gamma^\sigma\gamma^\rho\gamma^\nu\gamma^\tau)
\end{equation}

p.113-114 \\
\begin{equation}
  \Tr{\gamma^\mu\gamma^\nu} = 4g^{\mu\nu}
\end{equation}
\begin{equation}
  \Tr{\cancel{p}_1...\cancel{p}_n} = 0 \quad \forall\, n\, \in \, 2\Z+1
\end{equation}
\begin{equation}
  \Tr{\cancel{A}\cancel{B}\cancel{C}\cancel{D}} = 4(A\bdot B C \bdot D - A \bdot C B \bdot D + A \bdot D B \bdot C )
\end{equation}

p.146 \\
\begin{equation}
  \Tr{\gamma^\mu} = 0
\end{equation}
\begin{equation}
  \Tr{\gamma_5\cancel{p}_1...\cancel{p}_n} = 0 \quad \forall\, n\, \in \, 2\Z+1
\end{equation}
\begin{equation}
  \Tr{\gamma_5\gamma^\mu\gamma^\nu} = 0
\end{equation}
\begin{equation}
  \Tr{\gamma^\alpha\gamma^\beta\gamma^\gamma\gamma^\delta} = -4i\varepsilon^{\alpha\beta\gamma\delta}
\end{equation}

p.141-142
\begin{equation}
  \gamma_L = \frac{1}{2}(1-\gamma_5) \quad \implies \quad \gamma_L^2 = \gamma_L
\end{equation}
\begin{equation}
  \gamma_R = \frac{1}{2}(1+\gamma_5) \quad \implies \quad \gamma_R^2 = \gamma_R
\end{equation}
\begin{equation}
  \psi_L = \gamma_L\psi \quad \implies \quad \overline{\psi}_L = \overline{\psi}\gamma_R
\end{equation}
\begin{equation}
  \psi_R = \gamma_R\psi \quad \implies \quad \overline{\psi}_R = \overline{\psi}\gamma_L
\end{equation}

p.255 \\
\begin{equation}
  \psi^c = C\psi^T \quad \text{où} \quad C = i\gamma^0\gamma^2
\end{equation}
\begin{equation}
  (\psi_L)^c = \gamma_R\psi^c \quad \implies \quad \overline{(\psi_L)^c} = \overline{\psi^c}\gamma_L
\end{equation}

\section{Section efficace}
...

\section{Taux de désintégration calculés en classe}
...

\section{Règles de Feynman}
...

\section{Polarisations}
p.133 \\
\begin{equation}
  \sum_{polarisation} \varepsilon_\mu^*\varepsilon_\nu = - g_{\mu\nu} \quad \text{photon}
\end{equation}
\begin{equation}
  \sum_{polarisation} \varepsilon_\mu^*\varepsilon_\nu = - \left[g_{\mu\nu}- \frac{P_\mu P_\nu}{P^2}\right] \quad \text{W}
\end{equation}




\end{document}
